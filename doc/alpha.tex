\documentclass[11pt,a4paper]{article}
\usepackage{fullpage}
%%\usepackage{courier}
\usepackage{graphicx}
\usepackage{indentfirst}
\usepackage{array}
\newcolumntype{L}[1]{>{\raggedright\let\newline\\\arraybackslash\hspace{0pt}}m{#1}}
\newcolumntype{C}[1]{>{\centering\let\newline\\\arraybackslash\hspace{0pt}}m{#1}}
\newcolumntype{R}[1]{>{\raggedleft\let\newline\\\arraybackslash\hspace{0pt}}m{#1}}

\begin{document}

%TITLE
\begin{centering}
\begin{Large}
Docs for qt\_scripts 
\end{Large}
\end{centering}

\section{Table-like}
For full description of usage run the scripts with \texttt{--help} or \texttt{-h}

\subsection{prepare ONIOM inputs} 
\begin{tabular}{|R{2.5cm}|C{4.5cm}|C{1cm}|C{7cm}|}
    \hline
     & In & Out & Motivation \\
    \hline
    \texttt{amboniom.py} & \texttt{.prmtop} (or \texttt{.top}) \newline plus \newline
\texttt{.inpcrd} (or \texttt{.crd} / \texttt{.rst}) & \texttt{.com} &
Exact copy of forcefield parameters from Amber topology to gaussian.com file.
Specially useful with parameterized ligands and newer forcefields
(gaussian uses ff94 by default).\\
\hline
    \texttt{gfreezer.py} & \texttt{.com} & \texttt{.com}  & 
    Freeze protein further than a distance from high layer.
    Can be used to generate an external layer of water molecules.\\
\hline
\end{tabular}

\subsection{monitor and analysis}
\begin{tabular}{|r|C{2.5cm}|C{1.5cm}|C{8cm}|}
    \hline
     & In & Out & Motivation \\
    \hline
    \texttt{g2pdb.py} & \texttt{.log} / \texttt{.com} & \texttt{.pdb} &
Visualize scans and optimizations in VMD or Pymol by producing a .pdb file. \\
    \hline
    \texttt{hiig.py} & \texttt{.log} & & Get convergence situation of a job and a
 plot of energy along scan or optimization. \\
    \hline
    \texttt{gsurf.py} & any number of \texttt{.log} files & 3D plot & Plot energy 
against 2 reaction coordinates in a 3D plot. \\
    \hline
\end{tabular}

\subsection{restart jobs}
\begin{tabular}{|r|C{2.5cm}|C{1.5cm}|C{8cm}|}
    \hline
     & In & Out & Motivation \\
    \hline
    \texttt{gx.py} & \texttt{.log} & \texttt{.com} &
Restart calculation from a specific scan point (the last by default). \\
    \hline
    \texttt{gscan.py} & \texttt{.com} & \texttt{.com} & 
    Automatically set up scans by specifing final distance of
    reaction coordinate.\\
    \hline
    \texttt{paimei.py} &  &  & Automatically restarts error jobs.
Works well with scans and ground state optimizations. Read the manual for
    instructions. \\
    \hline
\end{tabular}



%%%%\section{Cheat Sheet}
%%%%
%%%%\subsection{prepare ONIOM inputs} 
%%%%
%%%%\texttt{amboniom.py --top (.top/.prmtop) --crd (.rst/.crd/.inpcrd)
%%%% --newgau new\_model.com}
%%%%
%%%%\vspace{3mm}
%%%%
%%%%\texttt{gfreezer.py model.com --keep 1000 -w -f 13}
%%%%
%%%%\subsection{monitor and analyze jobs}
%%%%
%%%%\texttt{g2pdb.py (.com/.log) -N -o -1 -s all -p newfile.pdb}
%%%%
%%%%\vspace{3mm}
%%%%
%%%%\texttt{hiig.py .log}
%%%%
%%%%\vspace{3mm}
%%%%
%%%%\texttt{gsurf.py scan1.log scan2.log opt3.log opt4.log -x 10 20 -y 30 31}
%%%%
%%%%\subsection{restart jobs}
%%%%
%%%%\texttt{gx.py scan.log -s 4 -o -1 --out scan2.com}
%%%%
%%%%\vspace{3mm}
%%%%
%%%%\texttt{gscan.py model.com --Bscan 10 20 --Bfreeze 30 31 --target 2.5 --delete}
%%%%
%%%%\vspace{3mm}
%%%%
%%%%\texttt{paimei.py}
%%%%
%%%%
%%%%
%%%%
%%%%
%%%%Generate a gaussian \textbf{.com} file from an amber topology (\textbf{.prmtop}
%%%%or \textbf{.top}) and coordinates (\textbf{.rst} or \textbf{.crd} 
%%%%or \textbf{.inpcrd}). The forcefield in the topology is fully preserved.
%%%%
%%%%Freeze residues based on distance \textbf{(-f option)} from high layer. 
%%%%Can remove all waters except the closest to protein atoms 
%%%%\texttt{(--keep option)}. More fancy options dedicated to waters available.


\section{Standardization of options}

This is yet to be implemented, if we all agree.

\begin{itemize}
    \item \texttt{-o} --- optimization steps
    \item \texttt{-s} --- scan steps
    \item \texttt{--out} --- filename for output of script
    \item \texttt{-e} --- energy type (oniom/scf/low)
    
\end{itemize}

%
%\section{Installation}
%
%\section{Building an ONIOM Model}
%
%\section{Monitoring and Analysis}
%
%\section{Restarting calculation}
%


\end{document}
